\chapter{MARCO TEÓRICO Y NORMATIVO}
    \section{Reglamentos de ejecución de proyectos}
    \section{Nivel actual de cobertura de servicio}
    \section{Sistemas de agua potable urbano y rural}
    \section{Periodo de diseño y estudios de población}
    \section{Dotación, consumo y almacenamiento}
        \subsubsection{Dotación y consumo de agua}
            La dotación de agua en un proyecto de abastecimiento se refiere a la cantidad de agua necesaria para satisfacer las necesidades de una determinada población o área geográfica. Este concepto es fundamental en la planificación y diseño de sistemas de suministro de agua potable.\\ 
            La dotación mínima a adoptarse debe ser suficiente para satisfacer los requerimientos de consumo: doméstico, comercial, industrial, social y público, así como considerar las pérdidas en la red de distribución \parencite{magne2008abastecimiento}:\\ 
            \begin{itemize}
                \item \textbf{Doméstico o residencial:} A esta categoría pertenecen aquellos suscriptores que utilizan el servicio exclusivamente para uso doméstico en la vivienda.
                \item \textbf{Social:} A esta categoría pertenecen aquellos predios utilizados para tareas de educación y salud (escuelas, colegios, puestos de salud), exclusivamente.
                \item \textbf{Oficial:} Esta categoría comprende instancias y áreas públicas no comprendidas para educación y salud, como son: jardines, parques, cuarteles, entidades del gobierno y otros.
                \item \textbf{Comercial:} Es la categoría a la cual pertenecen los suscriptores que utilizan el agua con fines de lucro dentro de alguna actividad comercial (restaurantes, lavado de vehículo, etc.).
                \item \textbf{Industrial:} Es la categoría a la cual pertenecen aquellos suscriptores que utilizan el agua para fines de lucro y en los que se lleva procesos industriales utilizándose el agua como insumo en el proceso de transformación (fábricas de vinos, chicherías, etc.).
            \end{itemize}
            
            El \textcite{os100} indica sino existiera estudios de consumo y no se justificara su ejecución, se considerara en sistemas con conexiones domiciliarias una dotación siguiente:\\ 
            Clima frío: 180 l/hab/d\\ 
            Clima Templados y Cálidos: 220 l/hab/d \\ 
            
            Para programas de vivienda con lotes de área menor o igual a $90m²$, las dotaciones serán de:\\ 
            Clima frío: 120 l/hab/d \\ Clima Templados y Cálidos: 150 l/hab/d \\ 
            
            Para sistemas de abastecimiento indirecto por surtidores para camión cisterna o piletas públicas se considera una dotación entre 30 y 50 l/hab/d respectivamente.\\ 

            Para habilitaciones de tipo industrial, deberá determinarse de acuerdo al uso en el proceso industrial, debidamente sustentado.\\ 

            Para habilitaciones de tipo comercial se aplicará la Norma IS.010 Instalaciones Sanitarias para Edificaciones.
        \subsubsection{Consumo de agua}
            
        \subsubsection{Almacenamiento del agua}
            
    \section{Fuentes de abastecimiento y sistemas de captación}
    \section{Variaciones de consumo}
        Según el \textcite{os100}, recomienda que los valores de las variaciones de consumo referidos al promedio diario anual deban ser fijados en base a un análisis de información estadística comprobada. Si no existieran los datos, se puede tomar en cuenta los siguiente:
